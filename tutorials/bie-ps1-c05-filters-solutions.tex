%%%%%%%%%%%%%%%%%%%%%%%%%%%%
% Způsob jak získat seznam příkazů
% sed -n '/\\begin{lstlisting}/,/\\end{lstlisting}/p' BI-PS1-Overview-20150106.tex  | gsed 's/\\$/\\n/'

%%%%%%%%%%%%%%%%%%%%%%%%%%%%%%%%%%%%%%%%%
% Beamer Presentation
% LaTeX Template
% Version 1.0 (10/11/12)
%
% This template has been downloaded from:
% http://www.LaTeXTemplates.com
%
% License:
% CC BY-NC-SA 3.0 (http://creativecommons.org/licenses/by-nc-sa/3.0/)
%
%%%%%%%%%%%%%%%%%%%%%%%%%%%%%%%%%%%%%%%%%

%----------------------------------------------------------------------------------------
%	PACKAGES AND THEMES
%----------------------------------------------------------------------------------------

%\documentclass{beamer}

 \documentclass[xcolor=table]{beamer}


\mode<presentation> {

% The Beamer class comes with a number of default slide themes
% which change the colors and layouts of slides. Below this is a list
% of all the themes, uncomment each in turn to see what they look like.

%\usetheme{default}
%\usetheme{AnnArbor}
%\usetheme{Antibes}
%\usetheme{Bergen}
%\usetheme{Berkeley}
%\usetheme{Berlin}
%\usetheme{Boadilla}
%\usetheme{CambridgeUS}
%\usetheme{Copenhagen}
%\usetheme{Darmstadt}
%\usetheme{Dresden}
%\usetheme{Frankfurt}
%\usetheme{Goettingen}
%\usetheme{Hannover}
%\usetheme{Ilmenau}
%\usetheme{JuanLesPins}
%\usetheme{Luebeck}
\usetheme{Madrid}
%\usetheme{Malmoe}
%\usetheme{Marburg}
%\usetheme{Montpellier}
%\usetheme{PaloAlto}
%\usetheme{Pittsburgh}
%\usetheme{Rochester}
%\usetheme{Singapore}
%\usetheme{Szeged}
%\usetheme{Warsaw}

% As well as themes, the Beamer class has a number of color themes
% for any slide theme. Uncomment each of these in turn to see how it
% changes the colors of your current slide theme.

%\usecolortheme{albatross}
%\usecolortheme{beaver}
%\usecolortheme{beetle}
%\usecolortheme{crane}
%\usecolortheme{dolphin}
%\usecolortheme{dove}
%\usecolortheme{fly}
%\usecolortheme{lily}
%\usecolortheme{orchid}
%\usecolortheme{rose}
%\usecolortheme{seagull}
%\usecolortheme{seahorse}
%\usecolortheme{whale}
%\usecolortheme{wolverine}

%\setbeamertemplate{footline} % To remove the footer line in all slides uncomment this line
%\setbeamertemplate{footline}[page number] % To replace the footer line in all slides with a simple slide count uncomment this line

%\setbeamertemplate{navigation symbols}{} % To remove the navigation symbols from the bottom of all slides uncomment this line
}

\usepackage{hyperref}


\usepackage{textcomp}
\usepackage{graphicx} % Allows including images
\usepackage{booktabs} % Allows the use of \toprule, \midrule and \bottomrule in tables

% Packages
\usepackage[czech, english]{babel}

\usepackage[utf8]{inputenc} % pro unicode UTF-8

\usepackage{verbatim}

\usepackage{xcolor}
\definecolor{backcolour}{rgb}{0.95,0.95,0.92}

\usepackage{listings}
\lstset{basicstyle=\ttfamily,
%	showspaces=false;
	showstringspaces=false,
	backgroundcolor=\color{backcolour},
%	commentstyle=\color{red},
%	keywordstyle=\color{blue}
}


% Macros
\newcommand\ti{\char`\~}                 % tilde char
\newcommand\s{\char`\/}                  % slah char	
%\newcommand\bs{\char`\\}             % backslah char
\newcommand\bs{\textbackslash}     % backslah char

\definecolor{links}{HTML}{2A1B81}
\hypersetup{colorlinks,linkcolor=,urlcolor=links}

%----------------------------------------------------------------------------------------
%	TITLE PAGE
%----------------------------------------------------------------------------------------

\title[BIE-PS1 - Tutorial 6]{Standard Input/Output, UNIX filters.} % The short title appears at the bottom of every slide, the full title is only on the title page

\author{Jan Trdlička} % Your name
\institute[FIT CTU] % Your institution as it will appear on the bottom of every slide, may be shorthand to save space
{
Department of Computer Systems\\
Faculty of Information Technology \\
Czech Technical University in Prague\\ % Your institution for the title page
\medskip
\textit{trdlicka@fit.cvut.cz} % Your email address
}
\date{\today} % Date, can be changed to a custom date

\begin{document}

\begin{frame}
\titlepage % Print the title page as the first slide
\end{frame}

\begin{frame}
\frametitle{Contents} % Table of contents slide, comment this block out to remove it
\tableofcontents % Throughout your presentation, if you choose to use \section{} and \subsection{} commands, these will automatically be printed on this slide as an overview of your presentation
\end{frame}

%----------------------------------------------------------------------------------------
%	PRESENTATION SLIDES
%----------------------------------------------------------------------------------------

%------------------------------------------------
\section{Standard Input/Output} 
\begin{frame}[fragile]
\frametitle{Standard Input/Output}
	\begin{itemize}

		\item  Where are the standard I/O of process  connected by default?
%BS		
		%\pause
		\begin{itemize}
			\item Standard input from keyboard.
			\item Standard output and standard error output to terminal.
		\end{itemize}
%ES

		%\pause
		\item How to redirect the standard I/O?
		
%BS		
		%\pause
		\begin{itemize}
			\item By symbols: \texttt{<}, \texttt{<<}, \texttt{>}, \texttt{>>},   \texttt{2>}, \texttt{2>>}, \texttt{n>\&m}, and \texttt{|}.
		\end{itemize}
%ES
		%\pause
		\item What is the meaning of file descriptors: 0, 1, and 2?		
%BS		
		%\pause
		\begin{itemize}
			\item 0 = stdin (standard input).
			\item 1 = stdout (standard output).
			\item 2 = sterr (standard error output).
		\end{itemize}
%ES

		%\pause
		\item Is the order of redirection important?
\begin{lstlisting}[language=bash]		
ls ~ foo 2>&1 >f1
ls ~ foo >f1 2>&1
\end{lstlisting}		
		
%BS		
		%\pause
		\begin{itemize}
			\item Yes.
		\end{itemize}
%ES

		%\pause
		\item How to discard the command error output?	
%BS		
		%\pause
		\begin{itemize}
		
			\item
\begin{lstlisting}[language=bash]		
ls . foo 2>/dev/null
\end{lstlisting}

		\end{itemize}
%ES

	\end{itemize}
\end{frame}

%------------------------------------------------
\begin{frame}[fragile]
\frametitle{Standard Input/Output}
	\begin{itemize}

		\item Is the following redirection valid for more lines?
\begin{lstlisting}[language=bash]		
ls ~ foo >std.out 2>std.err
ls ~ foo
\end{lstlisting}		
		
%BS		
		%\pause
		\begin{itemize}
			\item No.
			%\pause
			\item Permanent redirection can be made by command \texttt{exec}
\begin{lstlisting}[language=bash]		
exec >std.out 2>std.err
\end{lstlisting}				
		\end{itemize}
%ES

		%\pause
		\item What is the pipe?
	\end{itemize}
		
%BS	
	%\pause	
	\begin{figure}[h]
	 	\centering
        		\includegraphics[width=0.8\textwidth]{fig/pipe.png}
	\end{figure}
%ES

\end{frame}

%------------------------------------------------
\begin{frame}[fragile]
\frametitle{Standard Input/Output}
	\begin{itemize}

		\item How to redirect the standard output to the file \texttt{out.txt} and 
			the standard error output to the file \texttt{err.txt} for the following command?
			
		\begin{small}	
\begin{lstlisting}[language=bash]		
ls -la / foo
\end{lstlisting}
		\end{small}		
		
%BS		
		%\pause
		\begin{itemize}
			\item
\begin{lstlisting}[language=bash]		
ls -la / foo >out.txt 2>err.txt
\end{lstlisting}
		\end{itemize}
%ES

		%\pause
		\item How to append the standard output and the standard error output to the file \texttt{out.txt}  for the following command?
	
		\begin{small}		
\begin{lstlisting}[language=bash]		
ls -la / foo
\end{lstlisting}
		\end{small}	
		
%BS		
		%\pause
		\begin{itemize}
			\item
\begin{lstlisting}[language=bash]		
ls -la / foo >>out.txt 2>&1
\end{lstlisting}
		\end{itemize}
%ES

		%\pause
		\item How to count the number of lines  written on the standard output by the command \texttt{find /ect}?	

%BS		
		%\pause
		\begin{itemize}
			\item
\begin{lstlisting}[language=bash]		
find /etc | wc -l
\end{lstlisting}
		\end{itemize}
%ES		

		%\pause
		\item How to discard the error mesages from the previous solution?
				
%BS		
		%\pause
		\begin{itemize}
			\item
\begin{lstlisting}[language=bash]		
find /etc 2>/dev/null | wc -l
\end{lstlisting}
		\end{itemize}
%ES

	\end{itemize}
\end{frame}


%------------------------------------------------
\section{UNIX filters – overview} 
\begin{frame}[fragile]
\frametitle{UNIX filters – overview}
	\begin{itemize}

		\item  What is the meaning of the following filters?
				
		\begin{itemize}
			\item \texttt{tee},
			%\pause
			\item \texttt{split}, \texttt{cat},
			%\pause
			\item \texttt{head}, \texttt{tail},
			%\pause
			\item \texttt{cut}, \texttt{paste},
			%\pause
			\item \texttt{sort}, \texttt{uniq},
			%\pause
			\item \texttt{diff}, \texttt{patch},
			%\pause
			\item \texttt{cmp}, \texttt{comm}.
		\end{itemize}
		
	\end{itemize}
\end{frame}

%------------------------------------------------
\begin{frame}[fragile]
\frametitle{UNIX filters – overview}

%BS		
	\begin{figure}[h]
	 	\centering
        		\includegraphics[width=0.75\textwidth]{fig/filters.png}
	\end{figure}
%ES

\end{frame}

%------------------------------------------------
\section{UNIX filters} 
\begin{frame}[fragile]
\frametitle{UNIX filters}
	\begin{itemize}
	
		\item How to number all lines of manual page of the command \texttt{bash}?
		
%BS
		%\pause
		\begin{footnotesize}
\begin{lstlisting}[language=bash]		
man bash | cat -n
\end{lstlisting}
\begin{lstlisting}[language=bash]		
man bash | nl -ba
\end{lstlisting}
		\end{footnotesize}
%ES
 	
		%\pause
		\item How to number all lines of output of the command \texttt{/usr/sbin/useradd}?
		
%BS
		%\pause
		\begin{footnotesize}
\begin{lstlisting}[language=bash]		
/usr/sbin/useradd 2>&1 | cat -n
\end{lstlisting}
		\end{footnotesize}
%ES

		%\pause
		\item  How to number all lines of manual page of the command \texttt{bash} and 
			print only lines from 100 to 105 to the standard output?

%BS
		%\pause
		\begin{footnotesize}
\begin{lstlisting}[language=bash]		
man bash | cat -n | tail -n+100 | head -6   # Linux
man bash | cat -n | tail +100   | head -6   # Solaris
\end{lstlisting}

		%\pause
\begin{lstlisting}[language=bash]
man bash | cat -n | head -105 | tail -6
\end{lstlisting}

		\end{footnotesize}
%ES

		%\pause
		\item How to print only the number of lines of the file \texttt{/etc/passwd}?
			
%BS
		%\pause
		\begin{footnotesize}
\begin{lstlisting}[language=bash]		
wc -l </etc/passwd
\end{lstlisting}
		%\pause
\begin{lstlisting}[language=bash]		
wc -l /etc/passwd | cut -d' ' -f1
\end{lstlisting}
		\end{footnotesize}
%ES	
		
	\end{itemize}
\end{frame}

%------------------------------------------------
\begin{frame}[fragile]
\frametitle{UNIX filters}
	\begin{itemize}
	
		\item How to print the number of users currently logged in to the current host?
			
			\vspace{5mm} 
			Hints: 
			\begin{enumerate}
				\item Use command \texttt{finger} to get info about users currently logged. 
				\item Use command \texttt{users} to get info about users currently logged. 	
			\end{enumerate}		

%BS
			\vspace{5mm}
			Solutions:
			\begin{enumerate}
				%\pause
				\item
\begin{lstlisting}[language=bash]
finger | tail -n+2 | wc -l    # Linux	
finger -f | wc -l             # Solaris
\end{lstlisting}

				%\pause
				\item
\begin{lstlisting}[language=bash]		
users | wc -w
\end{lstlisting}

			\end{enumerate}
%ES
	\end{itemize}
\end{frame}

%------------------------------------------------
\begin{frame}[fragile]
\frametitle{UNIX filters}
	\begin{itemize}
	
		\item How to modify the previous solution so, that every user is counted only one times?
		
%BS
		\begin{enumerate}
			%\pause
			\item
\begin{footnotesize}
\begin{lstlisting}[language=bash]		
finger | tail -n+2 | cut -d' ' -f1 | \
sort -u | wc -l                        # Linux
\end{lstlisting}
\end{footnotesize}

			%\pause
			\item
\begin{footnotesize}
\begin{lstlisting}[language=bash]		
users | tr ' ' '\n' | sort -u | wc -l
\end{lstlisting}
\end{footnotesize}

		\end{enumerate}
%ES

		%\pause
		\item How to create alias \texttt{load}, which prints the number of users currently logged in to the current host (the previous solutions)
			 and the output should have the following format:

\begin{lstlisting}[language=bash,showspaces=true]		
User load:  13
\end{lstlisting}
	
%BS
		\begin{enumerate}
			%\pause
			\item
\begin{footnotesize}
\begin{lstlisting}[language=bash]
alias load='echo "User load:  $(finger | \
tail -n+2 | cut -d" " -f1 | sort -u | wc -l | \
tr -d " ")"'
\end{lstlisting}
\end{footnotesize}

			%\pause
			\item
\begin{footnotesize}
\begin{lstlisting}[language=bash]		
alias load='echo "User load:  $(users | \
tr " " "\n" | sort -u | wc -l | tr -d " ")"'
\end{lstlisting}
\end{footnotesize}

		\end{enumerate}
%ES

	\end{itemize}
\end{frame}

%------------------------------------------------
\begin{frame}[fragile]
\frametitle{UNIX filters}
	\begin{itemize}
	
		\item How to print the number of users, that have account on the local host?\\
		
		Hint: Use command \texttt{getent passwd} to get info about user accounts.

			
%BS
		\begin{itemize}
			%\pause
			\item
\begin{lstlisting}[language=bash]		
getent passwd | wc -l
\end{lstlisting}
		\end{itemize}
%ES

		%\pause
		\item How to save the sorted list of login names of users that have account on the local host to the file \texttt{list.txt} 
			and at the same time to print the number of these names to standard output?
			
%BS
		\begin{itemize}
			%\pause
			\item
\begin{footnotesize}
\begin{lstlisting}[language=bash]		
getent passwd | cut -d':' -f1 | sort | \
tee list.txt | wc -l
\end{lstlisting}
\end{footnotesize}
		\end{itemize}
%ES

		%\pause
		\item How to print only a name (the 5th column) of the user, that has the highest user ID (the 3rd column)?
			
%BS
		\begin{itemize}
			%\pause
			\item
\begin{lstlisting}[language=bash]		
getent passwd | sort -t3 -k3,3n | tail -1 | \
cut -d':' -f5
\end{lstlisting}
		\end{itemize}
%ES
			
	\end{itemize}
\end{frame}

%------------------------------------------------
\begin{frame}[fragile]
\frametitle{UNIX filters}
	\begin{itemize}
	
		\item How to print a frequency table "The number of processes per user",
			where the first column is the number of processes running by the user 
			and the second column is the user name. The table should be sorted by the number of processes in descending order.
		
			\vspace{5mm}
			Hint: Use command \texttt{ps -eo user} to get info about running processes on the current host. 		
%BS
		\begin{itemize}
			%\pause
			\item
\begin{lstlisting}[language=bash]		
ps -eo user | tail -n+2 | sort | uniq -c | \
sort -k1,1nr                            # Linux
\end{lstlisting}
		\end{itemize}
%ES

		%\pause
		\item How to print only names of the 3 largest items in the directory \texttt{/usr/bin}?			
%BS
		\begin{itemize}
			%\pause
			\item
\begin{lstlisting}[language=bash]		
ls -l /usr/bin | tail -n+2 | sort -k5,5n | \
tail -10 | tr -s ' ' | cut -d' ' -f9-   # Linux
\end{lstlisting}

			%\pause
			\item
\begin{lstlisting}[language=bash]	
ls -Sr /usr/bin | tail -3
\end{lstlisting}
		\end{itemize}
%ES
			
	\end{itemize}
\end{frame}

%------------------------------------------------
\begin{frame}[fragile]
\frametitle{UNIX filters}
	\begin{itemize}
	
		\item How to print a frequency table "The number of directories per group" of directory \texttt{/etc} (not recursively),
			where the first column is the number of directories owned by the group
			and the second column is the group name. The table should be sorted by the number of directories in ascending order.
		
			\vspace{5mm}
			Hint: Use command \texttt{ls -ld} to get info about the content of directory (the 4th column is the group name). 		
%BS
		\begin{itemize}
			%\pause
			\item
\begin{lstlisting}[language=bash]		
ls -ld /etc/*/ | tr -s ' ' | cut -d' ' -f4 | \
sort | uniq -c | sort -k1,1n
\end{lstlisting}

			%\pause
			\item
\begin{lstlisting}[language=bash]
stat --printf="%G\n" /etc/*/ | \
sort | uniq -c | sort -k1,1n
\end{lstlisting}

		\end{itemize}
%ES
			
	\end{itemize}
\end{frame}

%------------------------------------------------
\begin{frame}[fragile]
\frametitle{UNIX filters}
	\begin{itemize}
	
		\item How to copy files and directories, that are listed in the shell variable \texttt{LIST},
			to the directory, which name is saved in the file \texttt{Backup.txt} (you must create this directory first).\\
			The shell variable \texttt{LIST} contains filenames separated by colon (e.g.  \texttt{/tmp/a:/etc:/usr/bin:...})
			and filenames don't contain spaces and special characters. There are no aliases in the shell.
					
%BS

		\begin{itemize}
			%\pause
			\item
\begin{lstlisting}[language=bash]		
mkdir "$(cat Backup.txt)"
\end{lstlisting}

			%\pause
			\item
\begin{lstlisting}[language=bash]		
cp -r $(echo $LIST | tr ':' ' ') \
"$(cat Backup.txt)"
\end{lstlisting}
		\end{itemize}
%ES
			
	\end{itemize}
\end{frame}

%------------------------------------------------
\begin{frame}[fragile]
\frametitle{UNIX filters}
	\begin{itemize}
	
		\item The file \texttt{List.txt} contains the list of directories and it has the following structure:
		
		\begin{footnotesize}
\begin{lstlisting}[language=bash]		
Directory
---------
/etc/ssh
/bin
/usr/bin
...
\end{lstlisting}		
		\end{footnotesize}
		
		How to create the file \texttt{Top.txt}, that contains names of the 5 largest directories from \texttt{List.txt}?
					
%BS
		\begin{itemize}
			%\pause
			\item
\begin{lstlisting}[language=bash]		
du -s $(cat List.txt |tail -n+3) | \
sort -k1,1nr | head -5 | \ 
cut -d'TAB' -f2 >Top.txt
\end{lstlisting}

			%\pause
			\item To enter character TAB in bash, press CTRL+V and TAB (see \texttt{man bash}).
		\end{itemize}
%ES
			
	\end{itemize}
\end{frame}

%------------------------------------------------
\section{Homework} 
\begin{frame}[fragile]
\frametitle{Homework}
	\begin{itemize}
		\item Create alias \texttt{lss}, which prints names of files in the working directory sorted by file size.
%BS	
		%\pause
		\begin{itemize}
			\item
			\begin{footnotesize}
\begin{lstlisting}[language=bash]
alias lss='ls -al . | tail -n +2 | sort -k5,5n | \
tr -s " " | cut -d" " -f9 '
\end{lstlisting}
			\end{footnotesize}
		\end{itemize}	
%ES				
		
		\begin{comment}
		\item Create shell script that prints names of the 10 largest files (including their sizes),
			which are in your home directory and in its direct  subdirectories.		
		
		\item Patching of source code
		\begin{itemize}
			\item Create directory with at least 5 files. (e.g. C project: files \texttt{*.c}, \texttt{*.h}, \texttt{Makefile}, …)
			
			
			\item Fill the files with some text (e.g. output of command \texttt{ls}, \texttt{date}, \texttt{man}, …).
			
			\item Create copy of this directory.
			\item Modify some files in new directory. Remove/create some old/new one.
			\item Use command \texttt{diff} to compare both directories.
			\item Save the output of command \texttt{diff} to the file.
			\item Use the previous file in program \texttt{patch} to create from old directory structure new directory structure.
		\end{itemize}
		\end{comment}
		
	\end{itemize}
\end{frame}

%----------------------------------------------------------------------------------------

\end{document}